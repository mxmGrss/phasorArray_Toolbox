\documentclass{article}
\usepackage{amsmath}
\usepackage{graphicx}
\usepackage{hyperref}

\title{Documentation for PhasorSS Class}
\author{Phasor Toolbox}
\date{\today}

\begin{document}

\maketitle

\section{PhasorSS Class}

The \texttt{PhasorSS} class represents a state-space model with phasor arrays. It is designed to handle periodic systems and provides various methods for simulation and analysis.

\subsection{Properties}

The \texttt{PhasorSS} class has the following properties:

\begin{itemize}
    \item \texttt{A} - State matrix (PhasorArray)
    \item \texttt{B} - Input matrix (PhasorArray)
    \item \texttt{C} - Output matrix (PhasorArray)
    \item \texttt{D} - Feedthrough matrix (PhasorArray)
    \item \texttt{p} - LPV parameter for phase used for periodicity
    \item \texttt{isLPV} - Logical flag indicating if the system is LPV or LTV
    \item \texttt{T} - Period of the system if set to LTV
    \item \texttt{isReal} - Logical flag indicating if the system is real
    \item \texttt{StateName} - Cell array of state names
    \item \texttt{StateUnit} - Cell array of state units
    \item \texttt{InputName} - Cell array of input names
    \item \texttt{InputUnit} - Cell array of input units
    \item \texttt{InputGroup} - Structure of input groups
    \item \texttt{OutputName} - Cell array of output names
    \item \texttt{OutputUnit} - Cell array of output units
    \item \texttt{OutputGroup} - Structure of output groups
    \item \texttt{Name} - Name of the system
    \item \texttt{Notes} - Cell array of notes
    \item \texttt{UserData} - User-defined data
\end{itemize}

\subsection{Methods}

The \texttt{PhasorSS} class includes several methods for simulation and analysis:

\begin{itemize}
    \item \texttt{initial} - Perform initial condition simulation
    \item \texttt{step} - Perform step input simulation
    \item \texttt{impulse} - Perform impulse input simulation
    \item \texttt{lsim} - Simulate with input history and initial condition
    \item \texttt{feedback} - Feedback connection of two PhasorSS objects
    \item \texttt{toLPVss} - Convert to an LPV state-space system
    \item \texttt{toLTVss} - Convert to an LTV state-space system
\end{itemize}

\subsection{Usage Example}

To create a \texttt{PhasorSS} object, you can use the following example:

\begin{verbatim}
A = PhasorArray(cat(3,[0 1; -2 -3],[1/2 0; 0 0]),"z_pospart",true);
B = [0; 1];
C = [1 0];
D = 0;
T = 0.1;
Pss = PhasorSS(A, B, C, D, T, 'isReal', true);
\end{verbatim}

This creates a periodic real-valued state-space system with a specified period.

\end{document}