\documentclass{article}
\usepackage{amsmath}
\usepackage{graphicx}
\usepackage{hyperref}

\title{Phasor Toolbox Documentation}
\author{Your Name}
\date{\today}

\begin{document}

\maketitle

\section{Introduction}

The Phasor Toolbox is designed to facilitate the analysis and simulation of systems represented in the harmonic domain. This toolbox includes essential classes such as `PhasorSS` and `dSpaceDataExplorer`, which provide users with the tools necessary to model, simulate, and analyze periodic systems effectively.

\subsection{Purpose}

The primary purpose of the Phasor Toolbox is to offer a comprehensive framework for handling periodic matrices and state-space representations. By leveraging the capabilities of the `PhasorSS` class, users can create and manipulate state-space models that incorporate phasor arrays, enabling advanced simulations and analyses.

\subsection{Scope}

This documentation covers the following key components of the Phasor Toolbox:

\begin{itemize}
    \item Overview of the `PhasorSS` class, including its properties, methods, and usage examples.
    \item Detailed documentation of the `dSpaceDataExplorer` class, outlining its functionalities and integration within the toolbox.
    \item Guidelines for utilizing the toolbox effectively, including best practices and common use cases.
\end{itemize}

The subsequent chapters will delve into the specifics of each class, providing users with the necessary information to harness the full potential of the Phasor Toolbox.

\end{document}