\documentclass{article}
\usepackage{graphicx}
\usepackage{amsmath}
\usepackage{hyperref}

\title{dSpaceDataExplorer Class Documentation}
\author{Phasor Toolbox}
\date{\today}

\begin{document}

\maketitle

\section{Introduction}
The `dSpaceDataExplorer` class is designed to facilitate the exploration and analysis of data obtained from dSpace systems. This class provides a set of methods for loading, processing, and visualizing data, making it easier for users to interact with their datasets.

\section{Properties}
The `dSpaceDataExplorer` class includes the following properties:

\begin{itemize}
    \item \textbf{data}: A structure that holds the loaded data from dSpace.
    \item \textbf{time}: A vector representing the time stamps of the data.
    \item \textbf{signals}: A cell array containing the names of the signals available in the dataset.
\end{itemize}

\section{Methods}
The class provides several methods for data manipulation and visualization:

\subsection{loadData}
\texttt{loadData(filePath)} \\
Loads data from the specified file path.

\subsection{processData}
\texttt{processData()} \\
Processes the loaded data for analysis.

\subsection{visualizeData}
\texttt{visualizeData(signalName)} \\
Generates plots for the specified signal.

\section{Usage Example}
To utilize the `dSpaceDataExplorer` class, follow these steps:

\begin{verbatim}
% Create an instance of the dSpaceDataExplorer
explorer = dSpaceDataExplorer();

% Load data from a file
explorer.loadData('path/to/datafile.mat');

% Process the loaded data
explorer.processData();

% Visualize a specific signal
explorer.visualizeData('signalName');
\end{verbatim}

\section{Conclusion}
The `dSpaceDataExplorer` class is a powerful tool for users working with dSpace data, providing essential functionalities for data handling and visualization.

\end{document}